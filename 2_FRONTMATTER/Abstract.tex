\setcounter{page}{0}
\section*{Abstract}

Different solutions to solving blockchain scaling have been tried historically, usually by compromising on decentralisation. Ethereum has chosen to scale by switching to Proof of Stake consensus and adding data sharding to allow for Layer 2 execution to be cheaper. However, in the light of cross-domain MEV, even this strategy may have centralising forces built in.

In this work we review concepts relevant to the Maximal Extractable Value (MEV) and cross domain MEV. We use this knowledge to try and identify cross domain arbitrage. We do this by extracting Uniswap data from four different domains and analysing this dataset with two different methods. 

We were successful in identifying one smart contract, deployed to multiple domains, engaging in cross domain arbitrage. We also note the difficulties in quantifying cross domain MEV.

\newpage

% We argue why its detrimental to decentralisation.

% We discuss future implication of cross domain MEV.


% Scaling decentralised blockchains has traditionally been done by compromising on decentralisation.

% Exploring feasibility of detecting cross domain MEV with little bugdet? few resource? not industrial size resources?


% With the rise of staking providers and LSD's the threat seems greater than ever. !!!!!!!!!!!!!!!!!!!


% We present two different attempts at detecting cross domain MEV in a dataset