\section*{Dansk Resumé}
Forskellige løsninger på at skalere blockchains er blevet prøvet gennem tiden. Typisk går man på kompromis med decentralisering. Ethereum har valgt at skalere ved at skifte til Proof of Stake konsensus og tilføje data opdeling for at tillade 2. lags udførelse at blive billigere. Men selv denne løsning kan have centraliserende tendenser når man overvejer effekterne af kryds domæne maksimalt udtrækkelig værdi.

I dette værk kigger vi på relevante koncepter relateret til maksimalt udtrækkelig værdi (MUV) og kryds domæne MUV. Vi bruger denne viden til at prøve at finde kryds domæne arbitrage. Vi gør dette ved at hente Uniswap data fra fire forskellige domæner og derefter analysere dette datasæt med to forskellige metoder.

Vi fandt en smart contract, til stede på flere domæner, som lavede kryds domæne arbitrage. Vi noterer os også sværheden i at kvantificerer kryds-domæne MUV.

