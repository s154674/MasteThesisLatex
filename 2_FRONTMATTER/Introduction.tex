\chapter{Introduction}

In the last few years, awareness of Maximal Extractable Value (MEV) has grown steadily. The landscape has moved from Priority Gas Auctions to Flashbots bundles and beyond. There are still many open questions about its potential impact on decentralisation.  

As Ethereum's transition to Proof of Stake consensus approaches, these questions are more pressing than ever, with new concerns about Cross-Domain MEV \cite{crossdomainMEV} appearing. 

In this paper we seek to find out if Cross-Domain MEV is already here, and if we can find and quantify it. We do this by extracting data from the blockchain and analysing it for traces of Cross-Domain arbitrage. We also raise questions about its impact on decentralisation. 

The thesis is comprised of 6 main chapters. Chapter Two provides some context concerning the history of decentralised blockchains and Ethereum in particular. Then on to theory, illuminating key concepts in Chapter Three. Chapter Four provides details of the decision making and implementation of the software used to extract blockchain data. Chapters Five and Six looks at the results and discusses the implications of them. Finally we have a conclusion in Chapter Seven.